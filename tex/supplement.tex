%
% Copyright (c) Meta Platforms, Inc. and affiliates.
%
% This source code is licensed under the MIT license found in the LICENSE file
% in the root directory of this source tree. 
%

% SIAM Supplemental File Template
\documentclass[supplement,hidelinks,onefignum,onetabnum]{siamart220329}

\renewcommand{\headers}[2]{
\pagestyle{myheadings}
\markboth{\uppercase\expandafter{#2}}
{\uppercase\expandafter{#1}}}

%
% Copyright (c) Meta Platforms, Inc. and affiliates.
%
% This source code is licensed under the MIT license found in the LICENSE file
% in the root directory of this source tree. 
%

% SIAM Shared Information Template
% This is information that is shared between the main document and any
% supplement. If no supplement is required, then this information can
% be included directly in the main document.


% Packages and macros go here
\usepackage{lipsum}
\usepackage{amsfonts}
\usepackage{graphicx}
\usepackage{epstopdf}
%%%%\usepackage{algorithmic}
\ifpdf
  \DeclareGraphicsExtensions{.eps,.pdf,.png,.jpg}
\else
  \DeclareGraphicsExtensions{.eps}
\fi

% Add a serial/Oxford comma by default.
\newcommand{\creflastconjunction}{, and~}

% Used for creating new theorem and remark environments
\newsiamremark{remark}{Remark}
\newsiamremark{hypothesis}{Hypothesis}
\crefname{hypothesis}{Hypothesis}{Hypotheses}
\newsiamthm{claim}{Claim}

% Sets running headers as well as PDF title and authors
\headers{An efficient algorithm for integer lattice reduction}
        {Fran\c{c}ois Charton, Kristin Lauter, Cathy Li, and Mark Tygert}

% Title. If the supplement option is on, then "Supplementary Material"
% is automatically inserted before the title.
\title{An efficient algorithm for integer lattice reduction\thanks{Submitted
to the editors DATE.
\funding{This work was funded by Meta Platforms, Inc.}}}

% Authors: full names plus addresses.
\author{Fran\c{c}ois Charton\thanks{Meta Platforms, Inc.,
  6 Rue M\'{e}nars 75002, Paris, France (\email{fcharton@meta.com}).}
\and Kristin Lauter\thanks{Meta Platforms, Inc.,
  1101 Dexter Ave N, Seattle, WA (\email{klauter@meta.com}).}
\and Cathy Li\thanks{Meta Platforms, Inc.,
  1101 Dexter Ave N, Seattle, WA (\email{cathyli@meta.com}).}
\and Mark Tygert\thanks{Meta Platforms, Inc.,
  1 Facebook Way, Menlo Park, CA (\email{tygert@meta.com}).}}

\usepackage{amsopn}
\DeclareMathOperator{\nint}{nint}
\DeclareMathOperator*{\argmin}{argmin}
\DeclareMathOperator{\bigoh}{\mathcal{O}}
\def\Id{\rm Id}

%%% Local Variables: 
%%% mode:latex
%%% TeX-master: "ex_article"
%%% End: 


\externaldocument[][nocite]{paper}

% Optional PDF information
\ifpdf
\hypersetup{
  pdftitle={Supplementary materials:
            an efficient algorithm for integer lattice reduction},
  pdfauthor={Fran\c{c}ois Charton, Kristin Lauter, Cathy Li, and Mark Tygert}
}
\fi

\begin{document}

\maketitle



\section{Poorly performing alternatives}
\label{poor}

This supplementary section mentions two possible modifications that lack
the firm theoretical grounding of the algorithm presented in the main text
and performed rather poorly in numerical experiments.
These modifications may be natural, yet seem not to work well.
Subsection~\ref{multiple} considers adding to a basis vector
multiple other basis vectors simultaneously,
such that the full linear combination would minimize the Euclidean norm
if the coefficients in the linear combination did not have to be rounded
to the nearest integers.
Subsection~\ref{modified} considers a modified Gram-Schmidt procedure.

\subsection{Combining multiple vectors simultaneously}
\label{multiple}

One possibility is to choose a basis vector at random, say $a^i_j$,
and add to that vector the linear combination of all other basis vectors
which minimizes the Euclidean norm of the result, with the coefficients
in the linear combination rounded to the nearest integers.
That is, choose an index $j$ uniformly at random,
and calculate real-valued coefficients $c^i_{j,k}$ such that the Euclidean norm
$\| a^i_j - \sum_{k=1}^n c^i_{j,k} \cdot a^i_k \|$ is minimal,
where $c^i_{j,j} = 0$. Then, construct
$a^{i+1}_j = a^i_j - \sum_{k=1}^n \nint(c^i_{j,k}) \cdot a^i_k$.

Repeating the process for multiple iterations, $i = 0$, $1$, $2$, \dots,
would appear reasonable. However, this scheme worked well empirically only
when the number $n$ of basis vectors was very small, at least
when the number $m$ of entries in each of the basis vectors was equal to $n$.
Rounding the coefficients $c^i_{j,k}$ to the nearest integers is too harsh
for this process to work well.

\subsection{Modified Gram-Schmidt process}
\label{modified}

Another possibility is to run the classical Gram-Schmidt procedure
on the basis vectors, while subtracting off from all vectors
not yet added to the orthogonal basis the projections
onto the current pivot vector. In this modified Gram-Schmidt scheme,
each iteration chooses as the next pivot vector the residual basis vector
for which adding that basis vector to the orthogonal basis would
minimize the sum of the $p$-th powers of the Euclidean norms
of the reduced basis vectors. The iteration orthogonalizes the pivot vector
against all previously chosen pivot vectors and then subtracts off
(from all residual basis vectors not yet chosen as pivots)
the projection onto the orthogonalized pivot vector,
with the coefficients in the projections rounded to the nearest integers.

This scheme strongly resembles the LLL algorithm
of~\cite{lenstra-lenstra-lovasz}, but with a different pivoting strategy
(using modified Gram-Schmidt). Numerical experiments indicate that
the modified Gram-Schmidt performs somewhat similarly to
yet significantly worse than the classical LLL algorithm.
Omitting the bubble-sorting of the LLL algorithm
via the so-called ``Lov\'asz criterion'' spoils the scheme.

This scheme is also reminiscent of the variants of the LLL algorithm
with so-called ``deep insertions,'' as developed
by~\cite{schnorr-euchner}, \cite{fontein-schneider-wagner},
\cite{yasuda-yamaguchi}, and others.
LLL with deep insertions performs much better, however,
both theoretically and practically.
Other modifications to the LLL algorithm,
notably the BKZ and BKW methods reviewed by~\cite{nguyen-vallee} and others,
also perform much better than the modified Gram-Schmidt.



\section{Further figures}
\label{further}

This supplementary section presents figures analogous
to those of Section~\ref{results},
but using different values of the parameters $\delta$ and $p$
detailed in Subsection~\ref{figures}.
Figures~\ref{p2time1-1e-1}--\ref{p2err1-1e-1-31} are the same
as Figures~\ref{p2time1-1e-15}--\ref{p2err1-1e-15-31},
but with $\delta = 1 - 10^{-1}$ instead of $\delta = 1 - 10^{-15}$.
Figures~\ref{pstime1-1e-15}--\ref{pserr1-1e-1-31} are the same
as Figures~\ref{p2time1-1e-15}--\ref{p2err1-1e-15-31}
and Figures~\ref{p2time1-1e-1}--\ref{p2err1-1e-1-31} for $n = 192$,
but with varying values of $p$ rather than just $p = 2$.

\begin{figure}
\begin{centering}
{\includegraphics[width=0.495\textwidth]{../codes/1-1e-1/plot00t_lll.pdf}}
{\includegraphics[width=0.495\textwidth]{../codes/1-1e-1/plot00t_iterate.pdf}}

{\includegraphics[width=0.495\textwidth]{../codes/1-1e-1/plot10t_lll.pdf}}
{\includegraphics[width=0.495\textwidth]{../codes/1-1e-1/plot10t_iterate.pdf}}

\end{centering}
\caption{$\delta = 1-10^{-1}$, $p = 2$;
         the upper plots are for $q = 2^{13} - 1$,
         the lower plots are for $q = 2^{31} - 1$}
\label{p2time1-1e-1}
\end{figure}

\begin{figure}
\begin{centering}
{\includegraphics[width=0.495\textwidth]{../codes/1-1e-1/plot00frobmean1.pdf}}
{\includegraphics[width=0.495\textwidth]{../codes/1-1e-1/plot00frobmean2.pdf}}

{\includegraphics[width=0.495\textwidth]{../codes/1-1e-1/plot00minmean1.pdf}}
{\includegraphics[width=0.495\textwidth]{../codes/1-1e-1/plot00minmean2.pdf}}

\end{centering}
\caption{$\delta = 1-10^{-1}$, $p = 2$, $q = 2^{13} - 1$}
\end{figure}

\begin{figure}
\begin{centering}
{\includegraphics[width=0.495\textwidth]{../codes/1-1e-1/plot00frobmulti1.pdf}}
{\includegraphics[width=0.495\textwidth]{../codes/1-1e-1/plot00frobmulti2.pdf}}

{\includegraphics[width=0.495\textwidth]{../codes/1-1e-1/plot00minmulti1.pdf}}
{\includegraphics[width=0.495\textwidth]{../codes/1-1e-1/plot00minmulti2.pdf}}

\end{centering}
\caption{$\delta = 1-10^{-1}$, $p = 2$, $q = 2^{13} - 1$}
\end{figure}

\begin{figure}
\begin{centering}
{\includegraphics[width=0.495\textwidth]{../codes/1-1e-1/plot10frobmean1.pdf}}
{\includegraphics[width=0.495\textwidth]{../codes/1-1e-1/plot10frobmean2.pdf}}

{\includegraphics[width=0.495\textwidth]{../codes/1-1e-1/plot10minmean1.pdf}}
{\includegraphics[width=0.495\textwidth]{../codes/1-1e-1/plot10minmean2.pdf}}

\end{centering}
\caption{$\delta = 1-10^{-1}$, $p = 2$, $q = 2^{31} - 1$}
\end{figure}

\begin{figure}
\begin{centering}
{\includegraphics[width=0.495\textwidth]{../codes/1-1e-1/plot10frobmulti1.pdf}}
{\includegraphics[width=0.495\textwidth]{../codes/1-1e-1/plot10frobmulti2.pdf}}

{\includegraphics[width=0.495\textwidth]{../codes/1-1e-1/plot10minmulti1.pdf}}
{\includegraphics[width=0.495\textwidth]{../codes/1-1e-1/plot10minmulti2.pdf}}

\end{centering}
\caption{$\delta = 1-10^{-1}$, $p = 2$, $q = 2^{31} - 1$}
\label{p2err1-1e-1-31}
\end{figure}

\begin{figure}
\begin{centering}
{\includegraphics[width=0.495\textwidth]{../codes/1-1e-15/plot01t_lll.pdf}}
{\includegraphics[width=0.495\textwidth]{../codes/1-1e-15/plot01t_iterate.pdf}}

{\includegraphics[width=0.495\textwidth]{../codes/1-1e-15/plot11t_lll.pdf}}
{\includegraphics[width=0.495\textwidth]{../codes/1-1e-15/plot11t_iterate.pdf}}

\end{centering}
\caption{$\delta = 1-10^{-15}$, $n = 192$;
         the upper plots are for $q = 2^{13} - 1$,
         the lower plots are for $q = 2^{31} - 1$ \dots\
         the vertical ranges of the plots on the left are very small,
         with the vertical variations displayed
         being statistically insignificant, wholly attributable to randomness
         in the computational environment.}
\label{pstime1-1e-15}
\end{figure}

\begin{figure}
\begin{centering}
{\includegraphics[width=0.495\textwidth]{../codes/1-1e-15/plot01frobmean1.pdf}}
{\includegraphics[width=0.495\textwidth]{../codes/1-1e-15/plot01frobmean2.pdf}}

{\includegraphics[width=0.495\textwidth]{../codes/1-1e-15/plot01minmean1.pdf}}
{\includegraphics[width=0.495\textwidth]{../codes/1-1e-15/plot01minmean2.pdf}}

\end{centering}
\caption{$\delta = 1-10^{-15}$, $n = 192$, $q = 2^{13} - 1$}
\end{figure}

\begin{figure}
\begin{centering}
{\includegraphics[width=0.495\textwidth]{../codes/1-1e-15/plot01frobmulti1.pdf}}
{\includegraphics[width=0.495\textwidth]{../codes/1-1e-15/plot01frobmulti2.pdf}}

{\includegraphics[width=0.495\textwidth]{../codes/1-1e-15/plot01minmulti1.pdf}}
{\includegraphics[width=0.495\textwidth]{../codes/1-1e-15/plot01minmulti2.pdf}}

\end{centering}
\caption{$\delta = 1-10^{-15}$, $n = 192$, $q = 2^{13} - 1$}
\end{figure}

\begin{figure}
\begin{centering}
{\includegraphics[width=0.495\textwidth]{../codes/1-1e-15/plot11frobmean1.pdf}}
{\includegraphics[width=0.495\textwidth]{../codes/1-1e-15/plot11frobmean2.pdf}}

{\includegraphics[width=0.495\textwidth]{../codes/1-1e-15/plot11minmean1.pdf}}
{\includegraphics[width=0.495\textwidth]{../codes/1-1e-15/plot11minmean2.pdf}}

\end{centering}
\caption{$\delta = 1-10^{-15}$, $n = 192$, $q = 2^{31} - 1$}
\end{figure}

\begin{figure}
\begin{centering}
{\includegraphics[width=0.495\textwidth]{../codes/1-1e-15/plot11frobmulti1.pdf}}
{\includegraphics[width=0.495\textwidth]{../codes/1-1e-15/plot11frobmulti2.pdf}}

{\includegraphics[width=0.495\textwidth]{../codes/1-1e-15/plot11minmulti1.pdf}}
{\includegraphics[width=0.495\textwidth]{../codes/1-1e-15/plot11minmulti2.pdf}}

\end{centering}
\caption{$\delta = 1-10^{-15}$, $n = 192$, $q = 2^{31} - 1$}
\label{pserr1-1e-15-31}
\end{figure}

\begin{figure}
\begin{centering}
{\includegraphics[width=0.495\textwidth]{../codes/1-1e-1/plot01t_lll.pdf}}
{\includegraphics[width=0.495\textwidth]{../codes/1-1e-1/plot01t_iterate.pdf}}

{\includegraphics[width=0.495\textwidth]{../codes/1-1e-1/plot11t_lll.pdf}}
{\includegraphics[width=0.495\textwidth]{../codes/1-1e-1/plot11t_iterate.pdf}}

\end{centering}
\caption{$\delta = 1-10^{-1}$, $n = 192$;
         the upper plots are for $q = 2^{13} - 1$,
         the lower plots are for $q = 2^{31} - 1$ \dots\
         the vertical ranges of the plots on the left are very small,
         with the vertical variations displayed
         being statistically insignificant, wholly attributable to randomness
         in the computational environment.}
\label{pstime1-1e-1}
\end{figure}

\begin{figure}
\begin{centering}
{\includegraphics[width=0.495\textwidth]{../codes/1-1e-1/plot01frobmean1.pdf}}
{\includegraphics[width=0.495\textwidth]{../codes/1-1e-1/plot01frobmean2.pdf}}

{\includegraphics[width=0.495\textwidth]{../codes/1-1e-1/plot01minmean1.pdf}}
{\includegraphics[width=0.495\textwidth]{../codes/1-1e-1/plot01minmean2.pdf}}

\end{centering}
\caption{$\delta = 1-10^{-1}$, $n = 192$, $q = 2^{13} - 1$}
\end{figure}

\begin{figure}
\begin{centering}
{\includegraphics[width=0.495\textwidth]{../codes/1-1e-1/plot01frobmulti1.pdf}}
{\includegraphics[width=0.495\textwidth]{../codes/1-1e-1/plot01frobmulti2.pdf}}

{\includegraphics[width=0.495\textwidth]{../codes/1-1e-1/plot01minmulti1.pdf}}
{\includegraphics[width=0.495\textwidth]{../codes/1-1e-1/plot01minmulti2.pdf}}

\end{centering}
\caption{$\delta = 1-10^{-1}$, $n = 192$, $q = 2^{13} - 1$}
\end{figure}

\begin{figure}
\begin{centering}
{\includegraphics[width=0.495\textwidth]{../codes/1-1e-1/plot11frobmean1.pdf}}
{\includegraphics[width=0.495\textwidth]{../codes/1-1e-1/plot11frobmean2.pdf}}

{\includegraphics[width=0.495\textwidth]{../codes/1-1e-1/plot11minmean1.pdf}}
{\includegraphics[width=0.495\textwidth]{../codes/1-1e-1/plot11minmean2.pdf}}

\end{centering}
\caption{$\delta = 1-10^{-1}$, $n = 192$, $q = 2^{31} - 1$}
\end{figure}

\begin{figure}
\begin{centering}
{\includegraphics[width=0.495\textwidth]{../codes/1-1e-1/plot11frobmulti1.pdf}}
{\includegraphics[width=0.495\textwidth]{../codes/1-1e-1/plot11frobmulti2.pdf}}

{\includegraphics[width=0.495\textwidth]{../codes/1-1e-1/plot11minmulti1.pdf}}
{\includegraphics[width=0.495\textwidth]{../codes/1-1e-1/plot11minmulti2.pdf}}

\end{centering}
\caption{$\delta = 1-10^{-1}$, $n = 192$, $q = 2^{31} - 1$}
\label{pserr1-1e-1-31}
\end{figure}



\clearpage



\bibliographystyle{siamplain}
\bibliography{lattice}



\end{document}
